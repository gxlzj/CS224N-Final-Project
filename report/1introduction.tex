%!TEX root = report.tex

\section{Introduction}
%sentiment analysis
Sentiment analysis is one of the important task in natural language processing community[8] which help people navigate the huge amount of user-generated content available online. Machine learning systems that make decision on the attitude of viewpoints to be positive, neutral or negative that enable people to understand the enormous body of opinions on the Internet, ranging from product reviews to political positions. 

%social network
Interestingly, most of the viewpoints nowadays can be obtained from online website that actually has a social network behind it, since user-generated content often appears in the context of social media. Therefore nowadays user-relationship information is now more easily obtainable. For example, huge amount of tweets from Twitter express people's opinions on different subjects. Each tweet is associated with a user and users formed social network structure through the mechanisms of ``follower''. When a user forms a link in the network such as Twitter, they tend to have a personal relationship then the principle in language called ``homophily'' suggests that users who are connected via some social relationship may also share similar opinions or linguistic variation. Figure 1 from [1] gives an example of how users from different communities may understand the word ``sick'' differently.


In this paper, we are going to explore different methods that utilize social network information in sentiment analysis with deep learning. Network structure is useful and informative in NLP-related task as people within each community may have their own ``jargon'' in expressing ideas and sentiments. 
