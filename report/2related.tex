%!TEX root = report.tex
\section{Related Work}

\subsection*{Sentiment Analysis}

%cnn
The current state-of-the-art method is convolutional neural networks(CNN)[9] which takes word embeddings from sentences as inputs and output a softmax classification to identify sentence sentiment. The typical structure of such CNN is some convolutional layer on top of original sentence word embeddings, then a max pooling layer on top of the convolutional layer to extract some extreme information.  Finally, a dense layer with fully connected network is added to transform features from CNN to a softmax classifier. A simple CNN model with one convolutional layer of two-width window plus one max pooling layer with single channel can achieve amazing result[1]. Different initializations methods[11] could also be adopted to improve prediction accuracy, but they all share the similar structure as described above.


\subsection*{Network Node Embeddings}

To be added by Hao Yin ....


\subsection*{Aiding Classification with Social Network}

The intuition behind combing social network with classification is that users connected are more likely to hold similar opinions and use language similarly. 

Tan et al.(2011)[5] is the first paper to show social relationship information can be exploited to improve sentiment analysis. They have shown numerically that incorporating social-network information can indeed lead to statistically significant sentiment-classification improvements over the performance of a SVM baseline model that only has access to textual features. Yang and Eisenstein(2016)[1] is a more recent version for combining social network information and sentiment analysis. They study task for classifying sentiment to be positive, neutral and negative for each tweets given text and user ID information. Their model consider the author information and sentence information separately: each node (author) in the network is assigned an embedding vector using the LINE algorithm[2], and then is (softly) assigned each cluster on the network based on the embedding. Each cluster has its own model, which is a CNN model combined with max pool layer. Detail model specs are described in section 4 as a comparison to our model.

On the other hand, Yang and Chang(2016)[6] study another problems called entity linking, which is the task of identifying mentions
of entities in text, and linking them to entries in a knowledge base. They achieve the-state-of-art result with a tree-based model in Twitter data.  To further improve the performance, Yang et al.(2016)[7] propose to incorporate social network information in the same problem. Intuitively, socially linked individuals share interests, and are therefore likely to mention the same sorts of entities. Based on the previous the-state-of-art tree model[6], they build a bilinear model  that consider interactions of users and entities. This new model incorporating social network information has a F1 improvements of 1\%-5\% on benchmark datasets.